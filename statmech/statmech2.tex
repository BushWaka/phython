% THE canonical preamble
% wisdom from http://stackoverflow.com/questions/193298/best-practices-in-latex

% The Zen of LaTeX
%1. When writing a large document (book), keep the chapters in separate files
%2. Use a versioning system
%3. Repeated code (i.e. piece of formula occurring many times) is evil. Use macros
%4. Use macros to represent concepts, not to type less
%5. Use long, descriptive names for macros, labels, and bibliographic entries
%6. Use block comments
%   %===================================
%   to emphasize the beginning of sections and subsections
%7. Comment out suppressed paragraphs, don't delete them yet
%8. Don't format formulas (i.e. break them into many lines) until the final font size and page format are decided
%9. Learn to use BibTeX
%10. avoid eqnarray, use align, align*, or split (from amsmath)
%11. reading mathmode is a must http://www.ctan.org/tex-archive/info/math/voss/mathmode/
%12. If you really have to produce double-spaced documents, use setspace package instead of changing baselinestreach yourself.
%13. use \centering instead of \begin{center}
%14. use hyprref
%15. Use \newcommand to make things more logical.

\synctex=1
\documentclass[hyperpdf,bindnopdf]{article}
\usepackage{
nag, %It warns the user about the usage of old packages or commands (for example, using \it, \tt, etc.)
fixltx2e, %It fixes some 'mistakes' in Latex
booktabs, % must better-looking tables than latex default
amsmath,
cancel,
caption,
cleveref,
colortbl,
csquotes,
datatool,
multirow,
listings,
%graphicx,
pgfplots,
xcolor,
%mathpazo,
}
%fonts
%\usepackage[T1]{fontenc}
%\usepackage{lmodern }% enhanced version of the standard computer modern, actively developed
%\usepackage{cmbright}
%end fonts
%\usepackage[auto]{chappg} %page numbering by chapter
%\pagenumbering{bychapter}


%==============
%useful physics
%\usepackage{siunitx} % actively maintained
\usepackage{physymb} % by David Zaslavsky
%==============
%
%\usepackage{graphicx}
%\usepackage{setspace}
%\usepackage{fullpage}
%\usepackage{pdfpages}
%\usepackage{float}
%\usepackage{rotating}
%\usepackage{graphicx}
%\usepackage{setspace}
%\usepackage{fullpage}
%\usepackage{pdfpages}
%\usepackage{float}
%\usepackage{rotating}





\begin{document}
    \section{Omega and Z}
    \newcommand{\ThE}{thermal equilibrium}

        In this section, we want to answer why $Z$ is a more efficient technology than the good 'ol $\Omega$.
        \begin{align*}
            &P(\mu_i) = e^{-\beta E_i}
            \left( \frac{Z_{N-1}(\beta)}{Z_N(\beta)}\right)\\
            &P(\mu_1,\mu_2,\mu_3,\ldots) = \frac{e^{-\beta E_1} e^{-\beta E_2} e^{-\beta E_3} \cdots}{Z(\beta)}
                              = \frac{e^{-\beta E} }{Z(\beta)}\\
        \end{align*}

        %&=& &=& &=& &=& &=& &=& &=& &=& &=& =
        Let's zoom into the system to see what's really going in there.
        As usual, in \ThE, the probability of a microstate only depends on its energy.

            \[P(\mu_i) = \frac1{\Omega_1(E_{\mu_i})} \]
        \begin{align}
            P(\mu_1,\mu_2,\mu_3,\ldots) &= \frac1{\Omega_1(E_1)}
                                \frac1{\Omega_1(E_2)}
                                \frac1{\Omega_1(E_3)}
                                \ldots \\
            &= \frac1{\Omega_N(E)}
        \end{align}

        Notice that these equations already represent thermal equilibrium,
        since they are characterized only by energy.
        Thermal equilibrium is our one and only approximation\footnote{this includes the assumption that $\beta$ is constant in the range $E\pm\Delta E$},
        hence, one shouldn't expect to do more approximation in order to arrive at Z.\\
        There are three ways to exploit the relation between (1) and (2), but first I need to introduce several notations:

        %Preliminary
        \newcommand{\aE}{\epsilon}
        \newcommand{\soi}{\sum_i}
        \newcommand{\Stothei}[1] {{\hat\pdc}^{#1} S_1(\aE)}
        Equation (1) to (2) may look like a tiny step in \LaTeX,
        but it is actually a giant step.
        Rewrite $\Omega_1$ as $e^{ S_1}$, hence the relation becomes
        \begin{equation}
            \label{eq:mainassumption}
            \soi S_1(E_i) = S_N(E) = S_N\Big(\soi E_i\Big) \\
        \end{equation}


            \paragraph{1. principle of least assumption
            \footnote{a more mediocre and tedious derivation is given in the appendix}}
            For \eqref{eq:mainassumption} to be true,
            $S_N(E)$ must not depend on the spread of the $E_i$ distribution,
            where mathematically you can freely rearrange your energy distribution as you wish
            \[S_1(E_1) + S_1(E_2) = S_1(E_1+E_2) + S_1(0) \]
            From here we know that $S_1$ must be a linear function
            \[ S_1(x) = S_1(0) + \beta x\]
            where $\beta = \pdc S_1 = \mathrm{constant}$.


            Interpretations:
            \begin{itemize}
                \item Hey, the entropy is automatically maximized, too
                    (which means, it is redundant
                    to further assume that entropy is maximized).\\
                \item We can now write the probability as
                    \[ P(\mu_i) = e^{-S_1(0) -\beta E_i} \]
                \item any visual reasons on why the entropy increases linearly?
            \end{itemize}


            \paragraph{2. vary N}
            Back to one particle, we are now authorized to use this relation
            \begin{align*}
                &p(\mu_i) = \frac{\Omega_{N -1}(E-E_{\mu_i})}{\Omega_N(E)} \\
                &= \exp(\log\Omega_{N-1}(E-E_{\mu_i}) - \log \Omega_N(E))\\
                & = \frac{\Omega_{N-1}( E )}{\Omega_N(E)}
                    \exp( -\pd{}{E}\log \Omega_{N-1}E_i
                          - \frac12\pddc \log\Omega_{N-1}E_i^2
                          - \cdots
                          )\\
                & = \frac{\Omega_{N-1}( E )}{\Omega_N(E)}
                    \exp(-\beta E_i
                         -\frac12 \pd{\beta}{E} E_i^2
                         - \cdots  )\\
                & = \frac{\exp(-\beta E_i)}{Z_1(\beta,\mathcal{O}(E_i^2))}\\ 
            \end{align*}
            There are two completely different arguments
            to arrive at Boltzmann distribution:
            \begin{itemize}
                \item the multiplicity is maximized
                \item the state only depends on energy
            \end{itemize}

            Another way to look at thermal equilibrium:
            we randomly sample a small element of our system,
            and yet it will still look exactly the same
            as its parent system (redundancy of information, rescaling, whatever).

            Possible mathematical relations:
            \[Z_{1R}(\beta,E_i)  = \frac{\Omega_N(E)}{ \Omega_{N-1}(E)}
                             \frac1{\exp(+ \frac{\beta}{2C_v T}E_i^2
                                         - \cdots  )}\]
            Now we begin to see why we need log. All of our arguments
            \[\log Z_1 = \pd{}{N}\log \Omega_N(E)\]
            The difference between the two description is that
            the former gives us the real-life particle description,
            why the latter only tells us about the smoothed out representation of the system.

            In the Z description,
            we are instead viewing the system as a field (``N is not fixed'', but we don't care).\\
            %topic: http://ajp.aapt.org/resource/1/ajpias/v71/i11/p1136_s1


        \section{APPENDIX}
        \subsection{1. principle of least assumption: not even close to a harmonic oscillator}
        there are two styles to do the taylor expansion:
        
        \paragraph{1. expand around average energy \aE}
        \[ \soi S_1(\aE) +  \hat\pdc S_1 \soi(E_i-\aE) +  {\hat\pdc}^2 S_1 \soi (E_i-\aE)^2 + \cdots  = S_N(E) \]
        \[ NS_1(\aE) + \Stothei{} \cdot 0 + \Stothei{2}\sigma_\aE + \cdots = S_N(E)   \]
        \[ NS_1(\aE) + \beta \cdot 0 + \frac12\pd{\beta}{\aE}\sigma_\aE + \cdots = S_N(E)   \]
        Unfortunately it is no longer obvious whether the higher order terms will die or not. in general it could be such that, for example $ \pd{\beta}{\aE}\sigma_\aE = f(E)$.

        \paragraph{2. expand around $E=0$}
        \[ \soi S_1(0) +  \hat\pdc S_1(0) \soi E_i +  {\hat\pdc}^2 S_1(0) \soi E_i^2 + \cdots  = S_N(E) \]

         Now, here comes the crucial step in deriving $Z$, our \textbf{ second approximation: we truncate our series by claiming that  }(this would later means that the specific heat depends only on the variance of the energy!). This shouldn't be true for derivative with respect to microscopic variable $\pd{\beta}{\aE}=0$.
        %\[  \sum_{i=1}\sum_k \frac1{k!}\pdc^k_xS_1\Big|_{x=E_i} (E_i-E)^k = S_N(\sum_iE_i - NE) \]
        %\[  \sum_{i=1}S_1(E_i) + \sum_iS'_1(E_i)(E_i-E) + \cdots = S_N(0) \]




\end{document}

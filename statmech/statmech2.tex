% THE canonical preamble
% wisdom from http://stackoverflow.com/questions/193298/best-practices-in-latex

% The Zen of LaTeX
%1. When writing a large document (book), keep the chapters in separate files
%2. Use a versioning system
%3. Repeated code (i.e. piece of formula occurring many times) is evil. Use macros
%4. Use macros to represent concepts, not to type less
%5. Use long, descriptive names for macros, labels, and bibliographic entries
%6. Use block comments
%   %===================================
%   to emphasize the beginning of sections and subsections
%7. Comment out suppressed paragraphs, don't delete them yet
%8. Don't format formulas (i.e. break them into many lines) until the final font size and page format are decided
%9. Learn to use BibTeX
%10. avoid eqnarray, use align, align*, or split (from amsmath)
%11. reading mathmode is a must http://www.ctan.org/tex-archive/info/math/voss/mathmode/
%12. If you really have to produce double-spaced documents, use setspace package instead of changing baselinestreach yourself.
%13. use \centering instead of \begin{center}
%14. use hyprref
%15. Use \newcommand to make things more logical.

\synctex=1
\documentclass[hyperpdf,bindnopdf]{article}
\usepackage{
nag, %It warns the user about the usage of old packages or commands (for example, using \it, \tt, etc.)
fixltx2e, %It fixes some 'mistakes' in Latex
booktabs, % must better-looking tables than latex default
amsmath,
cancel,
caption,
cleveref,
colortbl,
csquotes,
datatool,
multirow,
listings,
%graphicx,
pgfplots,
xcolor,
%mathpazo,
}
%fonts
%\usepackage[T1]{fontenc}
%\usepackage{lmodern }% enhanced version of the standard computer modern, actively developed
%\usepackage{cmbright}
%end fonts
%\usepackage[auto]{chappg} %page numbering by chapter
%\pagenumbering{bychapter}


%==============
%useful physics
%\usepackage{siunitx} % actively maintained
\usepackage{physymb} % by David Zaslavsky
%==============
%
%\usepackage{graphicx}
%\usepackage{setspace}
%\usepackage{fullpage}
%\usepackage{pdfpages}
%\usepackage{float}
%\usepackage{rotating}
%\usepackage{graphicx}
%\usepackage{setspace}
%\usepackage{fullpage}
%\usepackage{pdfpages}
%\usepackage{float}
%\usepackage{rotating}





\usepackage{multicol}

\begin{document}

    \section{$\Omega$ and $Z$}
    \newcommand{\ThE}{thermal equilibrium}
        
        In this section, we want to answer why $Z$ is a more efficient technology than the good 'ol $\Omega$.
        \begin{align*}
            &P(\mu_i) = e^{-\beta E_i}
            \left( \frac{Z_{N-1}(\beta)}{Z_N(\beta)}\right)\\
            &P(\mu_1,\mu_2,\mu_3,\ldots) = \frac{e^{-\beta E_1} e^{-\beta E_2} e^{-\beta E_3} \cdots}{Z(\beta)} = \frac{e^{-\beta E} }{Z(\beta)}\\
        \end{align*}

        %&=& &=& &=& &=& &=& &=& &=& &=& &=& =
        Let's zoom into the system to see what's really going in there. As usual, in \ThE, the probability of a microstate only depends on its energy.

            \[P(\mu_i) = \frac1{\Omega_1(E_{\mu_i})} \]
        \begin{align}
            P(\mu_1,\mu_2,\mu_3,\ldots) &= \frac1{\Omega_1(E_1)} \frac1{\Omega_1(E_2)} \frac1{\Omega_1(E_3)} \ldots \\
            &= \frac1{\Omega_N(E)}
        \end{align}

        Notice that these equations already represent thermal equilibrium, since they are characterized only by energy and both the element and the whole system are described by the SAME characteristic function, $\Omega$. Thermal equilibrium is our one and only approximation, hence, one shouldn't expect to do more approximation in order to arrive at Z. Boltzmann relation means your system is renormalizable.\\
        There are three ways to exploit the relation between (1) and (2):
        \paragraph{1. not even close to a harmonic oscillator}
        Equation (1) to (2) may look like a tiny step in \LaTeX, but it is actually a giant step.
        Rewrite $\Omega_1$ as $e^{ S_1}$, hence the relation becomes
\[
            \sum_i S_1(E_i) = S_N(E) = S_N(\sum_i E_i) \\
            \]
        \newcommand{\aE}{\epsilon}
        which means $S_N$ must not depend on the spread of the $E_i$ distribution, to see this mathematically, let's taylor-expand around $\aE$
        \newcommand{\soi}{\sum_i}
        \newcommand{\Stothei}[1] {{\hat\pdc}^{#1} S_1(\aE)}
        \[ \soi S_1(\aE) +  \hat\pdc S_1 \soi(E_i-\aE) +  {\hat\pdc}^2 S_1 \soi (E_i-\aE)^2 + \cdots  = S_N(E) \]
        \[ NS_1(\aE) + \Stothei{} \cdot 0 + \Stothei{2}\sigma_E + \cdots = S_N(E)   \]
        or, $\Stothei{}=\beta$ where $\beta$ is a constant.
        %\[  \sum_{i=1}\sum_k \frac1{k!}\pdc^k_xS_1\Big|_{x=E_i} (E_i-E)^k = S_N(\sum_iE_i - NE) \]
        %\[  \sum_{i=1}S_1(E_i) + \sum_iS'_1(E_i)(E_i-E) + \cdots = S_N(0) \]

        Hey, the entropy is automatically maximized, too (which means, it is redundant to further assume that entropy is maximized).

        \paragraph{2. vary N}
        




        Back to one particle, we are now authorized to use this relation
        \begin{align*}
            &p(\mu_i) = \frac{\Omega_{N -1}(E-E_{\mu_i})}{\Omega_N(E)} \\
            &= \exp(\log\Omega_{N-1}(E-E_{\mu_i}) - \log \Omega_N(E))\\
            & = \frac{\Omega_{N-1}( E )}{\Omega_N(E)} \exp(-\pd{}{E}\log \Omega_{N-1}E_i - \frac{1}{2}\pddc \log\Omega_{N-1}E_i^2 - \cdots  )\\
            & = \frac{\Omega_{N-1}( E )}{\Omega_N(E)} \exp(-\beta E_i -\frac12 \pd{\beta}{E} E_i^2 - \cdots  )\\
            & = \frac{\exp(-\beta E_i)}{Z(\beta,\mathcal{O}(E_i^2))}\\ 
        \end{align*}
        where $\beta$ is just a label (until this point, you can safely assume that the). There are two completely different arguments: one, the multiplicty is maximized, two, the state only depends on energy

        Another way to look at thermal equilibrium: we randomly sample a small element of our system, and yet it will still look exactly the same as its parent system (redundancy of information, rescaling, whatever).

        Possible mathematical relations:
        \[Z_{1R}(\beta,E_i)  = \frac{\Omega_N(E)}{ \Omega_{N-1}(E)}\frac1{\exp(+ \frac{\beta}{2C_v T}E_i^2 - \cdots  )}\]
            Now we begin to see why we need log. All of our arguments
            \[\log Z_1 = \pd{}{N}\log \Omega_N(E)\]
        The difference between the two description is that the former gives us the real-life particle description, why the latter only tells us about the smoothed out representation of the system.

        In the Z description, we are instead viewing the system as a field (``N is not fixed'', but we don't care).\\
        topic: http://ajp.aapt.org/resource/1/ajpias/v71/i11/p1136_s1

\end{document}

% THE canonical preamble
% wisdom from http://stackoverflow.com/questions/193298/best-practices-in-latex

% The Zen of LaTeX
%1. When writing a large document (book), keep the chapters in separate files
%2. Use a versioning system
%3. Repeated code (i.e. piece of formula occurring many times) is evil. Use macros
%4. Use macros to represent concepts, not to type less
%5. Use long, descriptive names for macros, labels, and bibliographic entries
%6. Use block comments
%   %===================================
%   to emphasize the beginning of sections and subsections
%7. Comment out suppressed paragraphs, don't delete them yet
%8. Don't format formulas (i.e. break them into many lines) until the final font size and page format are decided
%9. Learn to use BibTeX
%10. avoid eqnarray, use align, align*, or split (from amsmath)
%11. reading mathmode is a must http://www.ctan.org/tex-archive/info/math/voss/mathmode/
%12. If you really have to produce double-spaced documents, use setspace package instead of changing baselinestreach yourself.
%13. use \centering instead of \begin{center}
%14. use hyprref
%15. Use \newcommand to make things more logical.

\synctex=1
\documentclass[hyperpdf,bindnopdf]{article}
\usepackage{
nag, %It warns the user about the usage of old packages or commands (for example, using \it, \tt, etc.)
fixltx2e, %It fixes some 'mistakes' in Latex
booktabs, % must better-looking tables than latex default
amsmath,
%amsthm, % Theorem Formatting
cancel,
caption,
cleveref,
colortbl,
csquotes,
datatool,
multirow,
listings,
%graphicx, % allows for eps images
pgfplots,
xcolor,
%mathpazo,
}
%fonts
%\usepackage[T1]{fontenc}
%\usepackage{lmodern }% enhanced version of the standard computer modern, actively developed
%\usepackage{cmbright}
%end fonts
%\usepackage[auto]{chappg} %page numbering by chapter
%\pagenumbering{bychapter}


%==============
%useful physics
%\usepackage{siunitx} % actively maintained
\usepackage{physymb} % by David Zaslavsky
%==============
%
%\usepackage{graphicx}
%\usepackage{setspace}
%\usepackage{fullpage}
%\usepackage{pdfpages}
%\usepackage{float}
%\usepackage{rotating}
%\usepackage{setspace}
%\usepackage{fullpage}
%\usepackage{pdfpages}
%\usepackage{float}
%\usepackage{rotating}




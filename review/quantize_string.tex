% THE canonical preamble
% wisdom from http://stackoverflow.com/questions/193298/best-practices-in-latex

% The Zen of LaTeX
%1. When writing a large document (book), keep the chapters in separate files
%2. Use a versioning system
%3. Repeated code (i.e. piece of formula occurring many times) is evil. Use macros
%4. Use macros to represent concepts, not to type less
%5. Use long, descriptive names for macros, labels, and bibliographic entries
%6. Use block comments
%   %===================================
%   to emphasize the beginning of sections and subsections
%7. Comment out suppressed paragraphs, don't delete them yet
%8. Don't format formulas (i.e. break them into many lines) until the final font size and page format are decided
%9. Learn to use BibTeX
%10. avoid eqnarray, use align, align*, or split (from amsmath)
%11. reading mathmode is a must http://www.ctan.org/tex-archive/info/math/voss/mathmode/
%12. If you really have to produce double-spaced documents, use setspace package instead of changing baselinestreach yourself.
%13. use \centering instead of \begin{center}
%14. use hyprref
%15. Use \newcommand to make things more logical.

\synctex=1
\documentclass[hyperpdf,bindnopdf]{article}
\usepackage{
nag, %It warns the user about the usage of old packages or commands (for example, using \it, \tt, etc.)
fixltx2e, %It fixes some 'mistakes' in Latex
booktabs, % must better-looking tables than latex default
amsmath,
cancel,
caption,
cleveref,
colortbl,
csquotes,
datatool,
multirow,
listings,
%graphicx,
pgfplots,
xcolor,
%mathpazo,
}
%fonts
%\usepackage[T1]{fontenc}
%\usepackage{lmodern }% enhanced version of the standard computer modern, actively developed
%\usepackage{cmbright}
%end fonts
%\usepackage[auto]{chappg} %page numbering by chapter
%\pagenumbering{bychapter}


%==============
%useful physics
%\usepackage{siunitx} % actively maintained
\usepackage{physymb} % by David Zaslavsky
%==============
%
%\usepackage{graphicx}
%\usepackage{setspace}
%\usepackage{fullpage}
%\usepackage{pdfpages}
%\usepackage{float}
%\usepackage{rotating}
%\usepackage{graphicx}
%\usepackage{setspace}
%\usepackage{fullpage}
%\usepackage{pdfpages}
%\usepackage{float}
%\usepackage{rotating}




\title{String Quantization}


\begin{document}
\maketitle 

\section{Discrete phonons}

Suppose we have a chain of masses m, each linked by a string of length a.
Each mass can be displaced about the equilibrium position vertically, by an amount
\(x_n\).

Then we have the following equations of motion:

\[m \ddot{x_n} = k \left(x_{n+1} + x_{n-1}\right) - 2 k x_n\]

We can work in the fourier basis, which decouples the motion of the masses into modes of vibrations labelled
by the wavenumber \(q \in (-\frac{\pi}{2}, \frac{\pi}{2})\):


\begin{align}
    & x_q = e^{i(qan-\omega(q) t)}\\
    & \omega_q^2 = -2k/m \cos(qa)- 2k/m
\end{align}

\subsection{Lagrangian and Hamiltonian Style}

It is nice to work in the Lagrangian and Hamiltonian scheme in preparation for quantization.
Let's set up a finite string of N masses, each mass m, separated by distance l in equilibrium. 
The displacement of the ith mass is \(\phi_i\).

The Hamiltonian is KE +PE:

\begin{align}
    & H = \frac12 \sum_{i = 0}^{N-1} ( m(\dot{\phi_i}^2) + k (\phi_i- \phi_{i-1})^2) \\
    & \frac12 \sum_{i = 0}^{N-1} l \left( \frac{m}{l} \dot{\phi_i}^2 + k l^2 (\phi_i - \phi_{i-1}/l)^2 \right)
\end{align}

The lagrangian is:

\begin{align}
    & L = \frac12 \sum_{i = 0}^{N-1} ( m(\dot{\phi_i}^2) - k (\phi_i- \phi_{i-1})^2) \\
    & \frac12 \sum_{i = 0}^{N-1} \left(l \frac{m}{l} \dot{\phi_i}^2 - k l^2 (\phi_i - \phi_{i-1}/l)^2 \right))
\end{align}


In the limit N goes to infinity, l to 0 , we can define the ratio: \(l/m = \mu\), \(kl = T\).
The lagrangian density, \(\mathcal{L} = L/l\):

\begin{align}
    & \mathcal{L} = \int dx  \frac12 \left(\mu \dot{\phi}^2 - T \pd{\phi}{x}^2 \right)
\end{align}


The principle of least action guarantees that the variation of the action is 0:

\newcommand{\La}{\mathcal L}
\newcommand{\Ha}{\mathcal H}

\begin{align}
    &\delta \int d^4 x \mathcal{L}(\phi, \partial_{x^\mu}{\phi}) = 0 \\
    \arr & \int d^4 x \pd{\La}{\phi} \delta \phi + \pd{\La}{(\partial_\mu \phi)} \delta \partial_\mu(\phi) \\
    \arr & \int d^4 (\pd{\La}{\phi} \delta - \partial_\mu \pd{\La}{(\partial_\mu \phi)}) \delta \phi 
    + \partial_\mu \left(\pd{\La}{(\partial_\mu \phi)} \delta \phi \right)
\end{align}

The last term is the 4-divergence, which can be converted to the difference between the spatial integrals
at the initial and final times, and therefore vanishes.

We therefore have the condition:

\begin{align}
    \partial_\mu \pd{\La}{(\partial_\mu \phi)} - \pd{\La}{\phi} = 0
\end{align}

Applying equation (11) to the string, we obtain the famous string equation:

\begin{align}
    \pdd{\phi}{x}-\frac{1}{v^2}\pdd{\phi}{t} = 0
\end{align}
where \(v^2 = \frac{T}{\mu}\)

\subsection{Classical Source Term}

\subsection{Canonical Momenta}

We can compute the following canonical momenta:

\begin{align}
    & \pd{L}{\dot{\phi}} = \mu \dot{\phi} - T \partial_x \phi v \\
    & \pd{L}{\partial_x \phi} = \frac{\mu}{v} \dot{\phi} - T \partial_x \phi
\end{align}


We see the canonical momenta is associated with the momentum of the transverse motion of the wave,
i.e. the larger \(\dot{\phi}\) is, the larger the canonical momentum.

\paragraph{Some redefinition}: for convenience purposes, we redefine the field \(\phi \arr \sqrt{T} \phi\),
so that T disappears from the equations. The field will have different units





\section{Decoupling the String into independent Oscillators}

\subsection{Normal Coordinates}

As with any wave equation, canonical quantization lends itself more nicely in a fourier basis.  We make the following guess:

\begin{align}
    & \phi_n = \frac{1}{\sqrt{L}} e^{i(k_n x- \omega_n t)}
\end{align}

Where we choose
\begin{itemize}
    \item \(\omega_n >0\)
    \item \(k_n = 2n \pi/L\) with \(n \in I\)(periodic Boundary Conditions)
    \item \(\omega_n^2 = v^2 k_n^2\) (Wave Equation)
\end{itemize}

We note that there are both positive and negative \(\omega and k_n\) solutions.
By Plancherel's theorem, the general solution can be written as follows.

\begin{align}
    & \phi(x,t)= \sum_{n= -\infty}^{\infty} c_n(a_n \phi_n + a_n* \phi_n*)
    & \phi(x,t)= \sum_{n = -\infty}^{\infty} c_n(a_n(t)e^{ik_n x} + a_n(t)* e^{-ik_n x})
\end{align}
Where \(a_n(t)= a(0)e^{-i \omega_n t}\)
We have indeed successfully decoupled the string into independent oscillators, since the coefficient \(a(t)\) satisfy:
\[\ddot{a_n}+ \omega_n^2 a_n = 0\]

The complex conjugate is there to ensure real solutions for the field.

We also have the following orthogonality identities:

\begin{align}:
    &\int_0^L \phi_n \phi_m = \delta_{n,-m} e^{-2i\omega_n t} \\
    &\int_0^L \phi_n* \phi_m = \delta_{n,m}
\end{align}

\subsection{Hamiltonian}

The new kinetic energy, potential energy and Hamitonian are:

\begin{align}
    & T = \frac{1}{2v^2} \int dx \dot{\phi}^2 \\
    & U = \frac{1}{2} \int dx \pd{\phi}{x}^2 \\
    & H = T + U = \sum_{-\infty}^{\infty} c_n^2(\frac{\omega^2}{v^2} + k_n^2) a_n^2(t) + c_n c_{-n} (k_n^2 - \frac{\omega_n^2}{v^2})Re(a_n a_{-n}) \\
\end{align}

Using the \(c_{-n} = -c_n\), we can rewrite:

\begin{align}
    & H = \sum_n 2 c_n^2 \frac{\omega_n^2}{v^2} a_n(t) a_n(t)*
\end{align}

We work in natural units, where \(h = c = 1\).  
We can make a dimensionless by defining \(c_n = \sqrt{\frac{v^2}{2 \omega_n}}\).
THe hamiltonian takes the form of:
\[H = \sum_n \omega_n a_n a*_n\]

Remarks:

\begin{itemize}
    \item The number a and a* have a lot of similarity to raising and lowering operators
    \item To make real a real dynamical variable, we can define \(q_n = \frac{1}{\sqrt{2 \omega_n}}(a_n + a_n*)\)
    \item We can then define \(p_n = \ud{q}{t} = - i (\frac{\omega_n}{2})^{1/2} (a_n - a_n*)\)
    \item The hamiltonian takes the form: \(H = \frac12 \sum_n (p_n^2 + \omega_n^2 q_n^2)\)
    \item Re-expressing \(q_n = \sqrt{2\omega_n} q_n + i \sqrt{\frac{2}{\omega_n}} p_n\)
    \item \(p_n = \sqrt{2\omega_n} q_n - i \sqrt{\frac{2}{\omega_n}} p_n\)
\end{itemize}



\section{Canonical Quantization}

We are now ready to quantize the string.  The canonical commutation relations are:

\begin{align}
    & [q_n, q_m] = [p_m, p_n] = 0 \\
    & [q_n, p_m] = i \hbar \delta_{nm}
\end{align}


Those commutation relations immediately imply commutation relations for the raising & lowering operators:

\[
     [a_n, a_m *]= \delta_{mn}
     \]





\end{document}
